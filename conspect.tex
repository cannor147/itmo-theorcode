\documentclass{article}
\usepackage[utf8]{inputenc}
\usepackage[russian]{babel}
\usepackage{amsfonts} 
\usepackage{graphicx}
\usepackage{indentfirst}
\usepackage[export]{adjustbox}[2011/08/13]
\usepackage{amsmath}
\usepackage{sectsty}
\sectionfont{\centering}
\makeatletter
\renewcommand*\env@matrix[1][*\c@MaxMatrixCols c]{%
  \hskip -\arraycolsep
  \let\@ifnextchar\new@ifnextchar
  \array{#1}}
\makeatother

\title{Теория кодирования. \\ Конспект практических занятий}
\author{Ерофей Башунов}
\date{}

\begin{document}

\maketitle

\newpage
\section*{3 сентября 2021}

\subsection*{Линейные коды}

Допустим, у нас есть некоторое векторное $n$-мерное пространство $\mathbb{F}^n_q$, где $q$ --- количество элементов. Выберем подпространство размерности $k$ ($n > k$), это и будет являться линейным кодом размерности $n$ и ранга $k$. Параметр $q$ отвечает за арность кода (при $q = 2$ код является бинарным).

Отображение из $k$ бит в $n$ бит --- добавление избыточной информации к векторам из подпространства. Это отображение необходимо для того, чтобы повысить устойчивость передачи информации к помехам путём добавления дополнительных ($n - k)$ <<проверяющих>> битов. Формально, оно записывается в виде $\mathbb{F}^k_q \rightarrow \mathbb{F}^n_q$.

Рассмотрим два вектора таких, что $a \in \mathbb{F}^k_q $ и $c \in \mathbb{F}^n_q$. Тогда преобразование можно рассматривать как умножение вектора на матрицу, а именно:

$$c = a \cdot \underset{k \times n}{G}$$

Матрица $G$ называется порождающей матрицей.

\subsection*{Проверочная матрица}

Проверочная матрица $H^T$ --- способ определить, является ли слово кодовым или нет. Необходимое и достаточное условие: $c \cdot H^T = 0$. Матрица находится с помощью уравнения:

$$\underset{k \times n}{G} \cdot \underset{n \times (n - k)}{H^T} = 0$$

Чтобы найти матрицу $H^T$, необходимо выполнить следующий алгоритм:
\begin{enumerate}
    \item Путём линейных преобразований, приводим матрицу $G$ к такому виду, чтобы в левой её части получилась единичная матрица. В процессе преобразований возможна перестановка столбцов (назовём матрицу перестановки $P$). Получится некоторая матрица $G' = G \cdot P$.
    \item Найдём матрицу $H'$ такую, что $G' \cdot H'^T = 0$. Пользуясь тем, что $G' = [I_k | S]$, получаем, что $H' = [-S^T | I_k]$
    \item Так как $(G \cdot P) \cdot (P^{-1} \cdot H^T) = G \cdot H^T = 0$ и $(G \cdot P) \cdot H'^T = G'^T \cdot H'^T = 0$, то верно равенство $H'^T = P^{-1} \cdot H^T$, следовательно $H = H' \cdot P$.
\end{enumerate}

\subsubsection*{Пример}

Допустим, у нас есть матрица $G$ вида

\begin{equation*}
\begin{pmatrix}
1 & 1 & 1 & 1 & 1 & 1 & 1 & 1 \\
1 & 1 & 1 & 1 & 0 & 0 & 0 & 0 \\
1 & 1 & 0 & 0 & 1 & 1 & 0 & 0 \\
1 & 0 & 1 & 0 & 1 & 0 & 1 & 0 \\
\end{pmatrix}
\end{equation*}

Путём линейных образований и перестановки 4 и 5 столбцов получаем матрицу $G'$:

\begin{equation*}
\begin{pmatrix}
1 & 0 & 0 & 0 & 1 & 1 & 1 & 0 \\
0 & 1 & 0 & 0 & 1 & 1 & 0 & 1 \\
0 & 0 & 1 & 0 & 1 & 0 & 1 & 1 \\
0 & 0 & 0 & 1 & 0 & 1 & 1 & 1 \\
\end{pmatrix}
\end{equation*}

Отсюда находим матрицу $H'$ как

\begin{equation*}
\begin{pmatrix}
0 & 0 & 0 & 1 & 1 & 0 & 0 & 0 \\
0 & 0 & 1 & 0 & 0 & 1 & 0 & 0 \\
0 & 1 & 0 & 0 & 0 & 0 & 1 & 0 \\
1 & 0 & 0 & 0 & 0 & 0 & 0 & 1 \\
\end{pmatrix}
\end{equation*}

А матрица $H$ будет иметь вид:

\begin{equation*}
\begin{pmatrix}
1 & 1 & 1 & 1 & 0 & 0 & 0 & 0 \\
1 & 1 & 0 & 0 & 1 & 1 & 0 & 0 \\
1 & 0 & 1 & 0 & 1 & 0 & 1 & 0 \\
0 & 1 & 1 & 0 & 1 & 0 & 0 & 1 \\
\end{pmatrix}
\end{equation*}

\subsection*{Логарифмическое отношение правдоподобия}

$$p(x) \cdot p(y | x) = p(x ^ y) = p(x | y) \cdot p(y)$$
$$p(x | y) = \frac{p(x) \cdot p(y | x)}{p(y)}$$

Выведем формулу логарифмического отношения правдоподобия. Помним, что $p(y | x) = \frac{1}{\sqrt{2 \pi \sigma^2}} \cdot \exp{-\frac{(y - x)^2}{2\sigma^2}}$

\begin{equation*}
\begin{gathered}
L = \log\frac{P(c_i = 0 | y_i)}{P(c_i = 1 | y_i)} = \log{\frac{\frac{1}{2} \cdot \frac{1}{\sqrt{2 \pi \sigma^2}} \cdot \exp{-\frac{(y_i + 1)^2}{2\sigma^2}}}{p(y_i)} \cdot \frac{p(y_i)}{\frac{1}{2} \cdot \frac{1}{\sqrt{2 \pi \sigma^2}} \cdot \exp{-\frac{(y_i - 1)^2}{2\sigma^2}}}} =\\= \log{\exp{\frac{(y_i - 1)^2 - (y_i + 1)^2}{2\sigma^2}}} = \frac{(y_i - 1)^2 - (y_i + 1)^2}{2\sigma^2} = -\frac{4 y_i}{2\sigma^2} = -\frac{2y_i}{\sigma^2}
\end{gathered}
\end{equation*}

\end{document}
